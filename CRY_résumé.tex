\documentclass{article}

\usepackage{a4wide}
\usepackage{titling}
\usepackage{geometry}
 \geometry{
 a4paper,
 total={150mm,260mm},
 left=20mm,
 top=20mm,
 }
\usepackage[version=3]{mhchem}
\usepackage{siunitx}
\usepackage{url}
\usepackage{graphicx}
\graphicspath{ {images/} }
\usepackage[rightcaption]{sidecap}
\usepackage{wrapfig}
\usepackage{amsmath}
\usepackage[utf8]{inputenc}
\usepackage{natbib}
\usepackage[english,french]{babel}

\bibliographystyle{unsrtnat}
\usepackage[pdftex]{hyperref}
\setlength\parindent{0pt}
\usepackage{xcolor}
\hypersetup{colorlinks,urlcolor=blue}
\usepackage{fancyhdr}
\usepackage{lastpage}
\usepackage[T1]{fontenc}
\usepackage{multicol}
\usepackage{amssymb}
\usepackage{ulem}
\usepackage{bm}




\fancyhf{}
\rfoot{Page \thepage \hspace{1pt}}   % //Page 1    %of \pageref{LastPage}}        // 1 of 1

\setlength{\columnseprule}{1pt}

\def\columnseprulecolor{\color{gray}}


\renewcommand{\labelenumi}{\alph{enumi}.}
\renewcommand{\ULdepth}{1.8pt}

%---------------------------- PAGE 1 --------------------------------------------------------------------

\title{}
\author{}
\date{}


\begin{document}

\begin{multicols*}{2}
[\section*{CRY-Résumé}]
\textbf{Définitions}\\
\uline{Adversaire passif:} peut espionner une comm. mais pas modifier le contenu\\
\uline{Adversaire actif:} peut espionner une comm., la modifier et se faire passer pour un des communiquants\\
\uline{Authenticité:} On peut prouver l'origine avec certitude\\
\uline{Intégrité:} On peut prouver la non-modification d'un message\\
\uline{Adversaire black-box:} Considère les algo comme des boîtes noires et les utilise comme oracle (\textit{attaque texte chiffré connu, clair connu, chiffré connu/choisi})\\
\uline{Adversaire gray-box:} Peut obtenir de l'info de l'implémentation \\
\textbf{Chance nbr à 100 chiffres soit premier?}\\
$ \frac{10^{100}}{\frac{10^{100}}{\ln(10^{100})}} = \frac{1}{\ln(10^{100})} = \frac{1}{10 \dot \ln(2)} $
\\
\noindent\rule{7cm}{0.5pt}
\textbf{Théorème Fermat-Euler}\\
$ a^{\varphi(n)} \equiv 1 \pmod n$ si \textbf{\textcolor{red}{a premier avec n}}\\
$7^{123456} \pmod{13} = 7^{10288 \cdot 12} \pmod{13}$\\
$= (7^{12})^{10288} \pmod{13} = (1)^{10288} \pmod{13} \equiv 1$\\
\noindent\rule{7cm}{0.5pt}
\textbf{Calculer un inverse mutliplicatif}\\
Après algo Euclide étendu:\\
$(3, -1, 1) \quad (1, -2, -7)$\\
Inverse de $13 \pmod{23} \rightarrow -7$\\
\noindent\rule{7cm}{0.5pt}
\textbf{Calculer log$_{3}$ de 4 mod 17}\\
On cherche x tel que $3^{x} = 4 \pmod{17}$\\
$\rightarrow$ On s'amuse à toutes les calculer...\\
\noindent\rule{7cm}{0.5pt}
\textbf{Factorisation d'un polynôme p}\\
$2 \leq Deg(p) \leq 3$: on regarde si racine\\
$Deg(p) > 3$ test si div. par polynôme Deg $\leq k/2$\\
\noindent\rule{7cm}{0.5pt}
\textbf{Anneau}\\
$\mathbb{F}[x]/m(x)$\\
\textbf{Corps}\\
$\mathbb{F}[x]/m(x)$ pour $m(x)$ irréductible sur $\mathbb{F}$\\
$GF(p)$ \text{: Corps de Galois premier}\\
$GF(p^{m})$ \text{: Corps de Galois non-premier}\\
$\rightarrow$ Si on demande si $GF(36)$ existe, non car il n'existe pas de paire (p,m) tel que $p^{m} = 36$\\
$\rightarrow$ Un corps de Galois à 81 éléments existe: $\mathbb{Z}_{3}$ avec un polynôme de \textbf{\textcolor{red}{irréductible}} de degré 4\\
\noindent\rule{7cm}{0.5pt}
\textbf{Montrer que $f$ est une permutation}\\
Il suffit de montrer que $f$ est inversible\\
\noindent\rule{7cm}{0.5pt}
\textbf{Chiffrement symétrique}\\
\underline{Modèles de sécurité}: pas trouver la clé décrypter le message, obtenir le moindre bit\\
\underline{Chiffrement par blocs}: \textit{input:} plaintext + key 
\textit{output:} ciphertext
\\
\underline{Chiffrement par flot}:\textit{input:} init. vector + key
\textit{output:} flot de bits (pour XOR avec plaintext)\\
\noindent\rule{7cm}{0.5pt}



\columnbreak


\textbf{Casser Vigenère}\\
Pour $(m, c = m+k) \rightarrow k = c-m$\\
\noindent\rule{7cm}{0.5pt}
\textbf{Casser Hill (avec n paires textes clairs)}\\
$ K = YX^{-1} \pmod m$ si $\det(X)$ premier avec m\\
\noindent\rule{7cm}{0.5pt}
\textbf{Opérations sur GF(2$^8$)}\\
\uline{Addition}: XOR\\
\uline{Multiplication par 0x02}: Shift vers la gauche (avec en plus un XOR avec 0x1b si carry)\\
\uline{Multiplication par 0x03}: Xor entre les multiplications par 0x02 et 0x01 \\
\textbf{CBC} $\leftarrow$ \textbf{\textcolor{blue}{Pas parallélisable}}\\
Si le nonce se répète:\\
On peut distinguer des messages qui commencent par les mêmes
blocs car les textes chiffrés correspondants seront les mêmes\\
\textbf{CTR} $\leftarrow$ \textbf{\textcolor{blue}{Parallélisable}}\\
Si le nonce se répète: \\
$C_{11} = M_{11} \oplus AES(NC_{1})$\\
$C_{12} = M_{12} \oplus AES(NC_{1})$\\
$\rightarrow C_{11} \oplus C_{12} = M_{11} \oplus M_{12}$ \\
\textbf{GCM}\\
Pareil que CTR mais propose l'authenticité.\\
L'AD permet d'authentifier sans les chiffré. Utile pour l'authentification de certaines valeurs dans les paquets réseau au moment du routage
\noindent\rule{7cm}{0.5pt}
\textbf{Taille des blocs}\\
\uline{AES}: 128 bits\\
\uline{DES}: 64 bits\\
\uline{Triple-DES}: 64 bits\\
\noindent\rule{7cm}{0.5pt}
\textbf{Paradoxe des anniversaires}\\
Pour une empreinte de \textit{l} bits, trouver une collision demande au plus $2^{\frac{l}{2}}$\\
$\rightarrow$ Taille moyenne du plaintext pour une répétition de blocs (AES, DES, Triple-DES): $k \cdot 2^{\frac{k}{2}}$\\
\noindent\rule{7cm}{0.5pt}
\textbf{Courbes elliptiques}\\
\uline{Inverse d'un point P:}\\
Pour un point $P = (x;y) \rightarrow P^{-1} = (x;-y)$\\
\textbf{Addition de points $P+Q$:}\\
$\mathcal{O}$ si $P = -Q$\\
$P$ si $Q =  \mathcal{O}$\\
$2P$ si $P = Q$ (Doublement de point)\\
sinon: Addition de points\\
\textbf{Formules pour $P(x_P; y_P) Q(x_Q; y_Q)$:}\\
\uline{Addition $x_P \not= x_Q$:}\\
	$x_R = (\frac{y_Q - y_P}{x_Q-x_P})^2 - x_P - x_Q$\\
	$y_R = -y_P + (\frac{y_Q - y_P}{x_Q-x_P})(x_P-x_R)$\\ 
\uline{Doublement de point:}\\
	$x_R = (\frac{3x_P^2 + a}{2y_P})^2 - 2x_P$\\
	$y_R = (\frac{3x_P^2 + a}{2y_P})(x_P-x_R) - y_P$\\
\textbf{Courbes elliptiques sur $GF(2^r)$}\\
\textcolor{red}{Il vaut mieux ne pas les utiliser}, des attaques rendent leur utilisation douteuse\\
\noindent\rule{7cm}{0.5pt}
\newpage

\textbf{Théorème de Hasse}\\
Si $N$ est le nombre de points sur une courbe elliptique définie sur un corps à q éléments, alors N est borné par:\\
$(q+1) -2\sqrt{q} \leqslant N \leqslant (q+1)+2\sqrt{q}$ \\
\noindent\rule{7cm}{0.5pt}
\textbf{Problème du logarithme discret}\\
\uline{Groupe multiplicatif:} $g^r$ \\
Trouver $r$ sachant $g$, $g^r$ est difficile\\
\uline{Groupe additif:} $rG$ \\
Trouver $r$ sachant $G$, $rG$ est difficile\\
\noindent\rule{7cm}{0.5pt}
\textbf{ECDH}\\
\\
\\
\\
\\
\\
\\
\\
\\
\\
\noindent\rule{7cm}{0.5pt}
\textbf{Théorème des restes chinois}\\
\\
$\bm{x \in \mathbb{Z}_{pq} \rightarrow (a;b) \in \mathbb{Z}_{p} \times \mathbb{Z}_{q}} $\\
$a = x \mod{p}$ et $b = x \mod{q} $\\

$\bm{(a;b) \in \mathbb{Z}_{p} \times \mathbb{Z}_{q} \rightarrow x \in \mathbb{Z}_{pq}} $\\
\small{$x \equiv (a(q^{-1} \mod{p})q + b(p^{-1} \mod{q})p \mod{pq} $}

\end{multicols*}






\end{document}


\nocite{}

\bibliography{bib}

\end{document}

